%
%	introduction.tex
%
%	Final Report Introduction
%
%	John Hughes and Michael Jean
%	University of Manitoba
%

\chapter{Introduction}

\section{Formula SAE}

Formula SAE\nomenclature{SAE}{Society of Automotive Engineers} is an engineering student design competition organized by the Society of Automotive Engineers dating back to 1978 \cite{fsaehistory}. Students from the University of Manitoba have participated in the competition almost every year since 1985. The competition consists of designing and constructing a small, open-wheeled, formula-style race car.

The Formula SAE vehicle is a performance car built with the primary goal of doing well in the dynamic events at the yearly competitions. These events test the vehicles' abilities in acceleration, braking, and handling. 

\section{Motivation}

Many of the issues that directly affect the teams' performance at competition relate to driver training, feedback, and the tuneability of the car. Most of the mechanical systems on the car must currently be imprecisely hand-tuned, and are packaged in hard to reach places, and require body panels or the seat to be removed for access.

Our overall goal is thus to improve the precision, adjustability, and repeatability of adjustment, of various important mechanical systems in the car, and to improve the efficiency of testing. We want a shorter driver feedback-tuning loop in order to eliminate overshoot and undershoot in tuning, and to avoid other external disturbances.

A second major goal is to improve upon the transmission control systems of previous years' designs.

\section{Problem Definition}

The goal of this work is to design and fabricate four electronic control modules to improve driver performance and improve the tuneability of the car. These modules include:

\begin{itemize}
\item a transmission and engine control module;
\item a brake bias adjustment module;
\item a wireless telemetry module; and
\item a driver interface module.
\end{itemize}

The transmission and engine control module should improve driver performance by reducing the time and effort required to change gears, and by allowing the engine to dynamically change operation to optimize available power. 

The brake bias adjustment module should improve tuneability by allowing the brake bias to be adjusted on the fly, even while the vehicle is in motion. It should reduce the need for manual adjustment and calibration.

The wireless telemetry module should improve vehicle performance and reliability by relaying critical vehicle operation data from various sensors to the pit crew, who can then make decisions regarding the various vechile parameters. The module should be able to broadcast to the pit area while the vehicle is competing in all of the dynamic events. It must operate with a minimum of data loss despite the distance and motion of the vehicle relative to the pit crew.

The driver interface should reduce driver effort and improve vehicle performance by allowing the driver to manage all the tuneable vehicle parameters from a simple interface located on the steering wheel, and by providing the driver with real-time feedback from the various vehicle sensors in an easy-to-read format. 

The components selected should not aversely affect the weight of the vehicle or dramatically increase the complexity of the mechanical systems, rather they should only require minimal mechanical adaptations to achieve the tuneability and performance goals outlined. As the team receives funding exclusively through 3rd-party sponsorship, the components selected should not dramatically increase the overall cost of the vehicle. 

\section{Strategy}

The design and implementation of each module requires several steps. 

Absolute requirements and specifications for system parameters must be estabilshed. The electromechanical interface linking the control systems and mechanical systems must be decided between our group and the mechanical engineers responsible for each respective system. The data requirements from the mechanical systems must be established and the appropriate sensors chosen.

Appropriate components for the electronic modules must be chosen, and circuits must be designed. The schematic of each module must be designed and an appropriate printed circuit board must be layed out. The boards must be manufactured and populated with components. Each board must be tested to ensure all wiring is correct and the functionality of all the components is correct.

Firmware for each module must be implemented once the hardware design is finished. Software libraries for various components must be written and tested. The libraries must be combined with a high-level control algorithm for each module. 

Each module must be tested in isolation to ensure proper functionality. The modules must then be interconnected and tested again.

\section{Outline of Thesis}

Outline goes here.
