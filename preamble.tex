%
%	preamble.tex
%
%	Final Report: Document Preamble
%
%	John Hughes and Michael Jean
%	University of Manitoba
%

\usepackage[T1]{fontenc}
%\usepackage[latin9]{inputenc}
\usepackage[utf8x]{inputenc}

\setcounter{secnumdepth}{3}
\setcounter{tocdepth}{3}

\usepackage{array}
\usepackage{babel}
\usepackage{fancyhdr}
\usepackage{float}
\usepackage[letterpaper]{geometry}
\usepackage{graphicx}
\usepackage{graphics}
\usepackage{lastpage}
\usepackage{tikz}
\usepackage{tikz-timing}[2009/05/15]
%\usetikztiminglibrary{nicetabs}
%\tikzset{timing/no nice tabs}
%\tikzset{timing/interval/normal}
%\tikzset{timing/nodes/old center}
\usepackage{subfigure}
\usepackage{tabularx}
\usepackage{array}
\usepackage{flafter,fnpos}
\usepackage{booktabs}
\usepackage{supertabular}

\usepackage{nomencl}
\makenomenclature

% Set Margins
\geometry{verbose,tmargin=1in,bmargin=1in,lmargin=1.5in,rmargin=1in}

%\setlength{\parskip}{12 pt}
\setlength{\parskip}{\medskipamount}
\makeatletter
\pagestyle{fancy}

\fancyhead[LE,LO]{Final Report}

% \fancyfoot[LE,LO]{
%   \tiny{
%     John Hughes, john\_hughes@umanitoba.ca\\
%     Michael Jean, michael.jean@shaw.ca
%   }
% }

%\fancyfoot[CE,CO]{\thepage\ of \pageref{LastPage}}
% \fancyfoot[RE,RO]{
%   \tiny{
%     \jobname{.tex}\\
%     \today
%   }
% }

\usetikzlibrary{shapes,arrows,shadows,calc}

% Define the arm and angle options
\def\myarm{1cm}
\def\myangle{0}
\tikzset{
  arm/.default=1cm,
  arm/.code={\def\myarm{#1}}, % store value in \myarm
  angle/.default=0,
  angle/.code={\def\myangle{#1}} % store value in \myangle
}

% Define the myncbar to path
\tikzset{
    myncbar/.style = {to path={
        % We need to calculate a couple of coordinates to help us draw
        % the path. 
        let
            % Same as (\tikztotarget)++(\myangle:\myarm)
            \p1=($(\tikztotarget)+(\myangle:\myarm)$)
        in
            -- ++(\myangle:\myarm) coordinate (tmp)
            % Find the projection of the (tmp) coordinate
            % on the line from the target to p1
            -- ($(\tikztotarget)!(tmp)!(\p1)$)
            -- (\tikztotarget)\tikztonodes
    }}
}

\tikzstyle{module} = [ draw=blue!50!black!50, % 50% red, 50% black
		       top color=white,
		       bottom color=blue!50!black!20,
		       rectangle,
		      very thick,
		      drop shadow,
]

\tikzstyle{block} = [ draw=red!50!black!50, % 50% red, 50% black
		      top color=white,
		      bottom color=red!50!black!20,
		      rectangle,
		      very thick,
		      minimum height=3em,
		      minimum width=6em,
		      inner sep=0.5em,
		      drop shadow,
		      text width=2cm,
		      text centered]
\tikzstyle{bus} = [coordinate]
\tikzstyle{input} = [coordinate]
\tikzstyle{output} = [coordinate]

\tikzstyle{small block} = [ draw=red!50!black!50, % 50% red, 50% black
		      top color=white,
		      bottom color=red!50!black!20,
		      rectangle,
		      very thick,
		      minimum height=3em,
		      minimum width=3em,
		      inner sep=0.5em,
		      drop shadow,
		      text width=1cm,
		      text centered]

\usepackage[
  unicode=true,
  pdfusetitle,
  bookmarks=true,
  bookmarksnumbered=true,
  bookmarksopen=false,
  breaklinks=false,
  pdfborder={0 0 0},
  backref=false,
  colorlinks=false
] {hyperref}