\documentclass[a4paper,10pt]{scrreprt}
\usepackage[utf8x]{inputenc}

% Title Page
\title{Final Report}
\author{}


\begin{document}
\maketitle

\begin{abstract}
\end{abstract}

\section{Engine Module}

\subsection{Electronics}

\subsubsection{Traction Cut Output}

The ECU allows a 0-5v analogue input to modify the allowable traction slip ratio from 0-100\% for the traction control.

The Engine module uses a simple SPI interfaced DAC from Texas Instruments, the TLV5623, to output the 0-5v analogue signal to the ECU.

The output voltage from the DAC is given by 

  \begin{equation}
    V_{out}=2\cdot{V_{ref}}\,\frac{Code}{2^n}\,[V]
  \end{equation} 

where $V_{ref}$ is the reference voltage input to the chip, $n=8\,(bits)$, and $Code$ is the digital input value ranging from $0$ to $2^{n-1}$. Since we want to output $5\,[V]$ at fullscale input,
  \begin{equation}
    2\cdot{V_{ref}}\,\frac{2^7}{2^8}=V_{ref}=5\,[V]
  \end{equation}.

\section{Driver Interface Module}

\subsection{Electronics}

\subsubsection{LCD Bias Circuit}

The LCD screen requires a large bias voltage of +22V.

A Linear Technology LT1615 step-up DC/DC converter was chosen as the centre of a boost converter circuit for the LCD.

\paragraph{Inductor Selection}

\begin{equation}
 L=\frac{V_{out}-V_{in(min)}+V_D}{I_{limit}}\cdot t_{off}
\end{equation}

with $V_D=0.4V$, $I_{limit}=350mA$, and $t_{off}=400ns$

\paragraph{Output Voltage}

To obtain a $V_{bias}$ of $+22V$, two resistors in the bias circuit provide a voltage divided feedback path from the output to the FB pin on the LT1615. The eqation relating the output voltage with the resistor values is
\begin{equation}
  R_1=R_2\cdot \left(\frac{V_{bias}}{1.23}-1\right)
\end{equation}
$R_1$ was chosen to be $2M\Omega$ to limit current flowing from the output to ground, and a suitable $R_1$ of $118k\Omega$ was found.

\paragraph{LCD Interface}
The LCD's 8-bit interface is suitable to be connected to the AT90's external memory interface. This way the LCD is a memory-mapped periferal, and all the control signals (Read and Write strobes, etc.), are handled by the memory controller.

The AT90's external memory interface uses Port A pins 0-7 as a multiplexed data and address bus which must be demultiplexed in order to offer seperate address and data busses. In operation, the external memory interface first puts out the address on the combined bus, followed by the data. The ALE (Address Latch Enable) signal signifies the difference.

In order to provide seperate address and data busses for the LCD controller, a fast octal D-Type latch from NXP was chosen to latch the address from the AT90. The width of the ALE pulse provided by the AT90, specified in the datasheet,
\begin{equation}
 t_{LHLL}=t_{CLCL}-15\,ns
\end{equation}
where $t_{CLCL}$ is the clock period. With the clock running at $16\,MHz$, $t_{LHLL}=48\,ns$.

The 74LVC373A latch from NXP requires a minimum LE pulse width of $4.5\,ns$, so is suitable as a demultiplexing interface.

The external memory on the AT90CAN128 starts at address 0x1100h, and there are two possible registers to read/write to on the LCD controller. The LCD controller therefore has it's single address select pin connected to the LSB of the address lines output from the latch. Since only two addresses are required, the upper 8 address lines of the external memory interface were not used.

A logical combination of the lower byte address lines will be connected to the CS (Chip Select) line on the LCD controller. Since the external memory controller only outputs control signals when the requested memory operation is in external space, it is safe to ignore the upper byte address lines.

It was chosen to tie the 2nd bit of the address lines to CS, which provides the following table of operations when interacting with the LCD controller:

\begin{table}
  \begin{center}
    \caption{Memory-mapped I/O operations with the LCD controller.}
    \begin{tabular}{|l|l|l|}
      \hline
      Address & Read Function & Write Function\\
      \hline
      \hline
      0x1102 & Status flag read & Display data and parameter write\\
      \hline
      0x1103 & Display dada and cursor address read & Command write\\
      \hline
    \end{tabular}
  \end{center}
\end{table} 

\end{document}          
