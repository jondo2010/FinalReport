%
%	system_architecture.tex
%
%	Final Report System Architecture
%
%	John Hughes and Michael Jean
%	University of Manitoba
%

\chapter{System Architecture}
\label{chap:architecture}

\section{Requirements}

\subsection{Telemetry Module}

The requirements for wireless data from the car were determined from experience gained in past years of testing and competitions. Data requirements can be broken down into two groups, data specific to the ECU and engine management, and all other run-time and performance data.

Data from the ECU is primarily the concern of the engine section head on the team. They are the most experienced with the tuning software and operation of the engine.

\paragraph*{DTASwin}

DTASwin is the engine management software provided by DTAFast for use with the ECU. All ECU configuration is done through the software, and all module parameters may be observed in real-time. The software interacts with the ECU through a proprietary protocol over an RS232 link.

An undocumented protocol feature required for the DTASwin software to function properly is that the software requires the Data Terminal Ready (DTR) signal to be pulled high when initially connecting to the ECU. Subsequent disconnections and reconnections after the software has already initialized do not have this requirement.


\paragraph*{ECU Data Throughput}

A linux serial sniffer utility


\paragraph*{RT Software}

\section{Theory of Operation}

The engine management data requires a duplex serial link between the ECU and a single laptop running the DTASwin software. Telemetry for this will be implemented using a point to point wireless transmission scheme. The address of the modem used by the engine management laptop will be programmed into the telemetry module ahead of time, and the address of the car's modem will be programmed into the driver running on the laptop.

During operation, data originating from the ECU will be received by the telemetry module, will be packetized, and will be sent over the modem specifically to the engine management laptop. Likewise, data originating from the DTASwin software will be packetized by the driver, and addressed to the car.

\section{Methodology}
\label{sec:methodology}

\section{Hardware Design}

\subsection{Telemetry Module}

\subsection{Wireless Modem}

To meet the range and data throughput requirements for the telemetry system, an \emph{XBee Pro} wireles modem was used.

The XBee requires 3.3v I/O levels and power supply, and so a second linear voltage regulator was used in the design, the LT1521 from Linear Technology.

Since the AT90CAN has only 2 built-in UARTS that were used for the RS232 interfaces to the ECU and DAQ, an third external UART was added to the design. The MAX3100 is a SPI-interfaced UART with an 8 word deep FIFO buffer. It is interfaced to the AT90's SPI pins and has an active-low IRQ line connected to external interrupt line EXT7.

%
\begin{table}
\centering{}\caption{XBee Pro Specifications}
\begin{tabular}{|l|l|}
\hline 
Outdoor line-of-sight  & Up to 1600m\tabularnewline
\hline 
Transmit Power  & $63\, mW$ ($18\, dBm$)\tabularnewline
\hline 
RF Data Rate  & $250,000\, bps$\tabularnewline
\hline
\end{tabular}
\end{table}

\subsection{Power Consumption}

During steady wireless transmission, current draw from the bench power supply remained around $200\, mA$.

\section{Software Design}

\subsection{Telemetry Module}

\section{Software}

The primary objective of the software running on the Telemetry Module is to push data around from various sources to various sinks. It uses the interrupt-driven USART drivers Mike wrote for both the on-board USARTs of the AT90, as well as the MAX3100 external USART extensively. These USART drivers are discussed in section ?? %TODO

\subsection{MAX3100 SPI UART driver}

An interrupt-driven driver was written to interface with the MAX3100 uart chip which allows asynchronous serial communication with the XBee modem on the Telemetry Module. The same circular buffer approach as with the internal USART drivers was taken, and in fact the drivers use very similar buffering code. Additionally, the software interfaces to the drivers were made as close as possible to allow interchangability by the main module software.

The datahseet for the MAX3100 states that the minimum SPI Clock period is $t_{CP}=238\, ns$, which results in a maximum frequency of approximately $4.2\, MHz$ \cite{MAX3100}. The SPI periferal was thus set to use 1/4 of the system clocks frequency resulting in $4\, MHz$.

The max3100 was configured to operate at 115200 baud which was matched with the XBee modem's interface rate. A simple throughput stress test was performed that continuously wrote data in 16 byte chunks into the max3100 transmit buffer. A fixed-length 32 byte buffer was never overrun with this test, and an actual data throughput of $8.251\, kB/s$ was observed with a logic analyser.


\subsection{libxbee}

Additionally, a driver to interact with the XBee modem through the MAX3100 was written by John. This driver, called \emph{libxbee} is available in Appendix \ref{cha:libxbee}. The XBee driver has two functions: 
\begin{enumerate}
\item to send and receive packetized data from the modem on behalf of the user software as per the XBee's API specification \cite{XBeeManual};
\item and to set the modem into API operation and manage the modem's operating state. 
\end{enumerate}
\emph{libxbee} implements the binary packet protocol described in the XBee manual, and is able to send and receive unicast and multicast packets. The driver is only capable of sending packets to 16-bit addresses. Full 64-bit address mode was not implemented as it was not needed in our 3 or 4 node network. The driver is also capable of sending command-type packets to the modem and reading the response.

Managing the modem's state was implemented as a state machine. Since minimal functionality was required, only a handful of AT commands were implemented. The driver is able to push the modem into API mode, after which all communication is done using the packet interface.

Similar to the rest of the software drivers implemented, \emph{libxbee} uses callbacks to provide the user software with incoming data. Function pointers to the USART api are also used, which makes it possible to reconfigure libxbee to use a different USART on-the-fly.

\subsection{Telemetry Operation}

\subsubsection{Buffering Issues}

Although initial average data throughput rates for both the DAC and the ECU were measured as part of the research done at the beginning of the project, non-trivial buffering issues did arise in the implementation of the telemetry module software. After several weeks of work spent debugging throughput problems with the module software, and with the aide of a protocol analyser, the root of the problem was tracked down. The AT90CAN microcontroller's interrupt vectors are fixed at the factory, therefore the priority of interrupts on the microcontroller is fixed. Unfortunately this lead to a resource starvation issue. The MAX3100 UART uses an external interrupt line, which has priority over the internal UART interrupts. When there is data in the MAX3100's transmit buffer, it will starve the internal UART interrupt to a certain extent. This was investigated by saturating the MAX3100 output buffer with a constant stream of data, and then inputting some bytes to the internal UART. Single input bytes were read properly, but any long stream of incoming bytes (more than 4 in a row) would cause hardware input buffer overruns in the built-in UART, and incoming data would be lost.

\begin{figure}[ht]
  \centering
  \label{fig:ecu_data}
  \begin{tikztimingtable} %[timing/nice tabs]
    $ECU_{Rx}$ & Z 10D{Polling sequence (7 bytes)} 14F Z \\
    $ECU_{Tx}$ & 12Z 6D{Reply sequence} ;[dotted] 2D{...}; 5D{(539 bytes)} Z\\
    \extracode
      %\tableheader{Signal}{Timing}
      \tablerules
  \end{tikztimingtable}
  \caption{ECU Serial Data Example}
\end{figure}

Based on this knowledge, we were able to classify conditions where the software would be unable to accurately process the incoming data. Unfortunately the method that the ECU transmits it's data falls within this characterization. When the ECU is communicating with the DTASWin software, the software will send a handful of bytes (around 7) to the ECU, and the ECU will reply with one large packet of approximately 540 bytes. If the MAX3100 is transmitting a large amount of data when this large packet comes in, bytes will be lost from the interrupt starvation issue.

Having determined the problem, several options lay before us to resolve the issue:
\begin{itemize}
  \item By lumping the ISR handlers for both the internal UART and the MAX3100 together, it would be possible to conditionally alter the priority of the processing to allow the incoming data to take precedence. This would however override the natural encapsulation that the two drivers had, and would require writing specialized drivers for use only on the telemetry module.
  \item Throttling the transmission of data to the MAX3100 by sequenced enabling and disabling of the interrupt could also reduce the starvation issue, but getting the sequencing and timing right would be difficult and more than likely result in further problems.
\end{itemize}

\chapter{Engine Module}

\label{chap:engine}


\section{Software Implementation}


\subsection{Clutch Solenoid PWM Output}

The clutch actuation system depends on being able to vary the duty
cycle of a fixed-frequency PWM signal fed to both clutch solenoids.
Table \textbackslash{}ref\{tab:pwm\_reqs\} lists the requirements
for the PWM generator. A PWM frequency of approximately $20\, Hz$
is the maximum operating frequency for the particular solenoid valves
we used. A duty cycle resolution of 1\textbackslash{}\% means the
minimum pulse width is $500\,\mu s$. Two channels are required for
operating both clutch valves.

%
\begin{table}
\caption{Clutch solenoid PWM requirements.\label{tab:pwm_reqs}}


\centering{}\begin{tabular}{|l|l|}
\hline 
PWM Frequency  & $f_{PWM}=20\, Hz$\tabularnewline
\hline 
PWM Resolution  & 1\% \tabularnewline
\hline 
Number of channels  & 2 \tabularnewline
\hline
\end{tabular}
\end{table}


An efficient method was devised to generate 2 synchronized PWM signals from the L9822E driver chip. The $32\, kHz$ external crystal was used as the input to the 8-bit TIM2 timer periferal on the microcontroller. The input to the timer was first scaled by 8 which provided a timebase:
\begin{equation}
TB=\frac{32.768\, kHz}{8}=4.096\, kHz
\end{equation}


The timebase period is then given by: \begin{equation}
\frac{1}{TB}\approx244\,\mu{S}\end{equation}


We then define a constant scaling factor for the PWM generator: \begin{equation}
{PWM\_DUTY\_SCALE}=\frac{T_{PWM}}{TB}\approx205\end{equation}


By loading the timers compare register with with a value scaled with the constant scaling factor, we can cause the timer to generate an interrupt precisely when a level change in the PWM is required.

When generating 2 waveforms with the same timer periferal, 8 combinations of duty cycles of channels A and B were identified, and can be seen in Figure \ref{fig:pwm_cases}.

Since the waveforms are synchronized, it can be seen that there are only 2 cases where 3 transitions per period are required, corresponding to \ref{fig:pwm_cases_1} and \ref{fig:pwm_cases_2}. This happens when both channels have $0<Duty<100\%$. 2 cases are also apparent when no transitions are required, shown in \ref{fig:pwm_cases_7} and \ref{fig:pwm_cases_8}.

An efficient generating routine was constructed to effect the level transitions and to reset the timer to interrupt at the next transition. For the two cases requiring 3 transitions per period, the timer interrupts 3 times per period. For the two cases where both channels have duty cycle between 0 and 100\%, the routine still interrupts once to allow for a change in duty cycle. For the rest of the cases described, the timer only interrupts twice per PWM period.

\begin{figure}[ht]
  \centering
  \subfigure[Case 1]
  {
    \label{fig:pwm_cases_1}
    \begin{tikztimingtable}
      $PWM_A$		& G 4H 4L \\
      $PWM_B$		& G 2H 6L \\
    \end{tikztimingtable}
  }
  \subfigure[Case 2]
  {
  \label{fig:pwm_cases_2}
  \begin{tikztimingtable}
    $PWM_A$		& G 4H 4L \\
    $PWM_B$		& G 6H 2L \\
  \end{tikztimingtable}
  }
  \subfigure[Case 3]
  {
  \label{fig:pwm_cases_3}
  \begin{tikztimingtable}
    $PWM_A$		& G 4H 4L \\
    $PWM_B$		& 8L \\
  \end{tikztimingtable}
  }
  \subfigure[Case 4]
  {
  \label{fig:pwm_cases_4}
  \begin{tikztimingtable}
    $PWM_A$		& G 4H 4L \\
    $PWM_B$		& G 8H G \\
  \end{tikztimingtable}
  }
  \subfigure[Case 5]
  {
  \label{fig:pwm_cases_5}
  \begin{tikztimingtable}
    $PWM_A$		& G 8H G \\
    $PWM_B$		& G 4H 4L \\
  \end{tikztimingtable}
  }
  \subfigure[Case 6]
  {
  \label{fig:pwm_cases_6}
  \begin{tikztimingtable}
    $PWM_A$		& 8L \\
    $PWM_B$		& G 4H 4L \\
  \end{tikztimingtable}
  }
  \subfigure[Case 7]
  {
  \label{fig:pwm_cases_7}
  \begin{tikztimingtable}
    $PWM_A$		& 8L \\
    $PWM_B$		& 8L \\
  \end{tikztimingtable}
  }
  \subfigure[Case 8]
  {
  \label{fig:pwm_cases_8}
  \begin{tikztimingtable}
    $PWM_A$		& G 8H G \\
    $PWM_B$		& G 8H G \\
  \end{tikztimingtable}
  }
  \caption{PWM Cases (1 period shown).}
  \label{fig:pwm_cases}
\end{figure}


\subsection{Clutch Control Scheme}


\section{Electronics}


\subsection{Traction Cut Output}

The ECU allows a 0-5v analogue input to modify the allowable traction
slip ratio from 0-100\% for the traction control.

The Engine module uses a simple SPI interfaced DAC from Texas Instruments,
the TLV5623, to output the 0-5v analogue signal to the ECU.

The output voltage from the DAC is given by

\begin{equation}
V_{out}=2\cdot{V_{ref}}\,\frac{Code}{2^{n}}\,[V]\end{equation}


where $V_{ref}$ is the reference voltage input to the chip, $n=8\,(bits)$,
and $Code$ is the digital input value ranging from $0$ to $2^{n-1}$.
Since we want to output $5\,[V]$ at fullscale input, \begin{equation}
2\cdot{V_{ref}}\,\frac{2^{7}}{2^{8}}=V_{ref}=5\,[V]\end{equation}
.

The DC input resistance $R_{in}$ on the traction cut input pin on
the ECU was measured using a series resistor with the input terminal
to be $R_{in}\approx155k\Omega$. The output current of the DAC therefore
will be at most $I_{out}=\frac{5v}{155k}\approx32.26\mu A$.

\chapter{Driver Interface Module}

\label{chap:driver}


\section{Electronics}


\subsection{Problems Encountered}
\begin{itemize}
\item Initial testing of the steering module showed an apparent short to
ground somewhere: current consumption was above 200mA at 12vdc supply.
It was discovered that the problem was that the tantalum capacitor
on the output side of the 5v LDO voltage regulator was backwards.
Neither of us knew beforehand that tantalum capacitors are actually
polarized. Flipping the capacitor fixed the high current draw problem.
\item The 4-bit level shifter was designed backwards, so that the 3.3v and
5v i/o pins were swapped. This was fixed by rotating the chip on the
board 180\textdegree{}, making 3 cuts, and using 5 airwires. 
\end{itemize}

\subsection{LCD Bias Circuit}

The LCD screen requires a large bias voltage of +22V.

A Linear Technology LT1615 step-up DC/DC converter was chosen as the
centre of a boost converter circuit for the LCD.


\paragraph{Inductor Selection}

\begin{equation}
L=\frac{V_{out}-V_{in(min)}+V_{D}}{I_{limit}}\cdot t_{off}\label{IndSel}\end{equation}


Using (\ref{IndSel}) with $V_{D}=0.4V$, $I_{limit}=350mA$, $t_{off}=400ns$,
$V_{out}=+22V$, and $V_{in(min)}=11.5V$ gives $L\approx12.45\mu{H}$.
The datasheet however suggests a value slightly smaller than calculated
should be suitable with only slight decrease in maximum output current.
Since the LCD requires very little current, we used an inductor value
of $10\mu H$.


\paragraph{Output Voltage}

To obtain a $V_{bias}$ of $+22V$, two resistors in the bias circuit
provide a voltage divided feedback path from the output to the FB
pin on the LT1615. The eqation relating the output voltage with the
resistor values is \begin{equation}
R_{1}=R_{2}\cdot\left(\frac{V_{bias}}{1.23}-1\right)\end{equation}
 $R_{1}$ was chosen to be $2M\Omega$ to limit current flowing from
the output to ground, and a suitable $R_{1}$ of $118k\Omega$ was
found.


\subsection{LCD Data Interface}

The LCD's 8-bit interface was suitable to be connected to the AT90's
external memory interface. This way the LCD becomes a memory-mapped
periferal to the AT90, and all the control signals (Read and Write
strobes, etc.), are handled by the memory controller.

The AT90's external memory interface uses Port A pins 0-7 as a multiplexed
data and address bus which must be demultiplexed in order to offer
seperate address and data busses. In operation, the external memory
interface first puts out the address on the combined bus, followed
by the data. The ALE (Address Latch Enable) signal signifies the difference
\cite{AT90CAN}.

In order to provide seperate address and data busses for the LCD controller,
a fast octal D-Type latch from NXP was chosen to latch the address
from the AT90. The width of the ALE pulse, $t_{LHLL}$, provided by
the AT90, is specified in the datasheet as \begin{equation}
t_{LHLL}=t_{CLCL}-15\, ns\end{equation}
 where $t_{CLCL}$ is the clock period. With the clock running at
$16\, MHz$, $t_{LHLL}=48\, ns$.

The 74LVC373A latch from NXP requires a minimum LE pulse width of
$4.5\, ns$, so is suitable as a demultiplexing interface.

The external memory on the AT90CAN128 starts at address 0x1100h, and
there are two possible registers to read/write to on the LCD controller.
The LCD controller therefore has it's single address select pin connected
to the LSB of the address lines output from the latch. Since only
two addresses are required, the upper 8 address lines of the external
memory interface were not used.

A logical combination of the lower byte address lines will be connected
to the CS (Chip Select) line on the LCD controller. Since the external
memory controller only outputs control signals when the requested
memory operation is in external space, it is safe to ignore the upper
byte address lines.

It was chosen to tie the 2nd bit of the address lines to CS, which
provides the following table of operations when interacting with the
LCD controller:

%
\begin{table}
\begin{centering}
\caption{Memory-mapped LCD Interface}

\par\end{centering}

\centering{}\begin{tabular}{|l|l|l|}
\hline 
Address  & Read Function  & Write Function\tabularnewline
\hline
\hline 
0x1102  & Status flag read  & Display data and parameter write\tabularnewline
\hline 
0x1103  & Display dada and cursor address read  & Command write\tabularnewline
\hline
\end{tabular}
\end{table}
